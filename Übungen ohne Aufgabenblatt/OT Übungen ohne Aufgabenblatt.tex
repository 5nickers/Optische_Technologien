\input{header.tex}


\begin{document}

\maketitle

Dieser Text ist unter dieser \href{http://creativecommons.org/licenses/by-nc-sa/4.0/}{Creative Commons} Lizenz veröffentlicht.

\textcolor{red}{Ich erhebe keinen Anspruch auf Vollständigkeit oder Richtigkeit. Falls ihr Fehler findet oder etwas fehlt, dann meldet euch bitte über den Emailkontakt.}

\tableofcontents


\newpage



\section{Aufgabe 1}


Wir betrachten eine Glasscheibe die mit einer Antireflexschicht aus $MgF_2$ bedampft ist. Die Schicht hat einen Brechungsindex von $n_{AR} = 1,38$ und wirkt gegen die Wellenlänge $\lambda = \unit[550]{nm}$. Zunächst bestimmen wir die Dicke der Antireflexschicht:

\begin{align*}
d &= \frac{\lambda_0}{4 \cdot n_{AR}} = \frac{550 \cdot 10^{-9}}{4 \cdot 1,38} = \unit[99,6]{nm}
\intertext{Diese Schicht erzeugt einen Ganunterschied von $\frac{\lambda}{2} = \unit[275]{nm}$. Der Phasenunterschied ist dann:}
\phi &= \frac{2 \pi}{\lambda} \cdot 275 = \frac{\pi \cdot 550}{\lambda}
\intertext{Mit dem Wissen, das $I \sim \cos^2 \left( \frac{\phi}{2} \right)$ ist, können wir den Verlauf der Intensität abhängig von der Phasenverschiebung darstellen. Beispielhaft habe ich eine Tabelle gebaut:}
\end{align*}

\begin{center}
	\begin{tabular}{|c|c|c|}
		 $\lambda$ & $\phi$ & $I$ \\ 
		 \hline
		400  &  &  \\ 
		$\vdots$  &  &  \\ 
	    700	 &  &  \\ 		
	\end{tabular} 
\end{center}

Wenn man das plottet, dann erhält man einen Graphen , der bei $\unit[550]{nm}$ ein Minimum haben sollte.

Nun schauen wir uns noch an wie gut die Antireflexionsschicht das reflektierte Licht reduziert.


\begin{figure}[h]
	\begin{minipage}[hbt]{5cm}
		\centering
		\includegraphics[width=5cm]{A1_1.jpg}
	\end{minipage}
	\hfill
	\begin{minipage}[hbt]{7cm}
		\centering
		\includegraphics[width=7cm]{A1_2.jpg}
	\end{minipage}
\end{figure}


Bei dem Glas ohne Reflexionsschicht erhalten wir nach $I \sim E^2$:

\begin{align*}
0,2^2 = 0,04 = \unit[4]{\%}
\intertext{Bei Glas mit Reflexionsschicht gilt für den Reflexionsgrad:}
R &= \frac{1,38 - 1}{1,38 + 1} = 0,159 = \unit[16]{\%} = E_0
\intertext{Die intensität ist dann:}
A_{eff} &= 2 \cdot E_0 \cdot \cos \left( \frac{\phi}{2} \right) = 32 \cdot \cos \left( \frac{\phi}{2} \right) \\
I &\sim A_{eff}^2 = 4 \cdot E_0^2 \cdot \cos^2 \left( \frac{\phi}{2} \right)
\end{align*}

Wenn wir uns die Phasenverschiebung mit $\frac{\pi \cdot 550}{\lambda}$ anschauen, dann erkennen wir, dass sich die Phasenverschiebung zu kleinen Wellenlängen stärker ändert als zu großen.










\end{document}