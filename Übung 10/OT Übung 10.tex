\input{header.tex}


\begin{document}

\maketitle

Dieser Text ist unter dieser \href{http://creativecommons.org/licenses/by-nc-sa/4.0/}{Creative Commons} Lizenz veröffentlicht.

\textcolor{red}{Ich erhebe keinen Anspruch auf Vollständigkeit oder Richtigkeit. Falls ihr Fehler findet oder etwas fehlt, dann meldet euch bitte über den Emailkontakt.}

\tableofcontents


\newpage



\section{Aufgabe 1}


\subsection*{a)}

Für den Lichtstrom gilt allgemein:

\begin{align*}
\phi &= K(\lambda) \cdot V(\lambda) \cdot \phi_e \qquad \text{mit} \qquad K = K_m = \unit[683]{lm/W}
\intertext{Da nur \unit[2]{\%} am Auge ankommen rechen wir dies runter:}
\phi_e &= 500 \cdot 0,02 = \unit[10]{W} \\
\Rightarrow \phi &= K_m \cdot V(500nm) \cdot 10 = 683 \cdot 0,35 \cdot 10 = \unit[2390,5]{lm}
\end{align*}


\subsection*{b)}

\begin{align*}
I_e &= \frac{\d \phi_e}{\d \Omega} = \frac{\phi_e}{4 \pi} = \frac{500}{4 \pi} = \unit[39,78]{W/sr} \\
I &= \frac{\phi}{4 \pi} = \frac{2390,5}{4 \pi} = \unit[190,22]{lm/sr}
\end{align*}


\subsection*{c)}

\begin{align*}
M_e &= \frac{\phi_e}{A} = \frac{500}{50 \cdot 10^{-4}} = \unit[10^5]{W/m^2}
\end{align*}


\subsection*{d)}

Wir betrachten einfach eine Kugel mit dem Radius $r = \unit[2]{m}$.

\begin{align*}
E_e &= \frac{\d \phi_e}{\d A} = \frac{500}{4 \pi r^2} = \frac{500}{4 \pi \cdot 2^2} = \unit[9,95]{W/m^2} \\
E &= 683 \cdot 0,35 \cdot 9,95 = \unit[2378]{lm/m^2}
\end{align*}


\subsection*{e)}

\begin{align*}
\phi_e &= 9,95 \cdot A_{Loch} = 9,95 \cdot \pi \cdot 0,05^2 = \unit[0,0195]{w} \\
\phi &= K \cdot V(\lambda) \cdot \phi_e = 683 \cdot 0,35 \cdot 0,0195 = \unit[4,66]{lm}
\end{align*}


\section{Aufgabe 2}

\subsection*{a)}

\begin{align*}
V(441) &\approx 0,05 \qquad q_e = \unit[10]{mW} \\
V(633) &\approx 0,3 \qquad q_e = \unit[4]{mW} 
\intertext{Damit ergeben sich relative Helligkeiten von:}
\phi_{441} &= 683 \cdot 0,05 \cdot 10^{-2} = \unit[0,34]{lm} \\
\phi_{633} &= 683 \cdot 0,3 \cdot 4 \cdot 10^{-3} = \unit[0,82]{lm}
\end{align*}


\subsection*{b)}

\begin{align*}
V(488) &\approx 0,25 \\
V(543) &\approx 0,8 
\intertext{Nun soll gelten, das die Helligkeiten gleich sein sollen. Dazu setzen wir gleich:}
\phi(488) &= \phi(543) \\
\Leftrightarrow 683 \cdot 0,25 \cdot \phi_{Ar} &= 683 \cdot 0,8 \cdot 0,5 \cdot 10^{-3} \\
\Leftrightarrow \phi_{Ar} &= \frac{0,8 \cdot 0,5 \cdot 10^{-3}}{0,25} = \unit[1,6]{mW}
\end{align*}


\section{Aufgabe 3}

Zuerst eine Zeichnung zur Erläuterung:

\begin{figure}[h]
	\centering
	\includegraphics[scale=0.1]{A3_1.jpg}
\end{figure}


Wir wissen das die Beleuchtungsstärke so definiert $\frac{lm}{m^2}$, wir versuchen dies nun anders darzustellen:

\begin{align*}
\frac{lm}{m^2} &= \frac{lm}{sr} \cdot \frac{sr}{m^2} \\
\intertext{Wir bestimmen nun den letzten Teil des Terms:}
\Delta \Omega &\approx \frac{A}{r^2} = \frac{1}{3^2} = \unit[\frac{1}{9}]{sr/m^2}
\intertext{Jetzt können wir einfach umstellen und ausrechnen:}
E &= \frac{1}{9} \cdot I \\
\Leftrightarrow I &= 9 \cdot E = 9 \cdot 100 = \unit[900]{lm/sr}
\end{align*} 

\subsection*{b)}

Wir berechnen hier den Abschwächungsfaktor, der einen Meter neben dem Punkt $P$ anzuwenden ist:

\begin{figure}[h]
	\centering
	\includegraphics[scale=0.1]{A3_2.jpg}
\end{figure}


\begin{align*}
r' &= \sqrt{3^2 + 1^2} = \sqrt{10} \\
\theta &= \arctan \left( \frac{1}{3} \right) = 18,434
\intertext{Der Abscwächungsfaktor ist dann:}
x &= \frac{r^2}{r'^2} \cdot \cos(\theta) = \frac{9}{10} \cdot \cos(18,434) = 0,853 \\
\Rightarrow E' &= 100 \cdot x = 85,3
\end{align*}






















\end{document}