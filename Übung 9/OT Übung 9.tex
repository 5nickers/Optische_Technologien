\input{header.tex}


\begin{document}

\maketitle

Dieser Text ist unter dieser \href{http://creativecommons.org/licenses/by-nc-sa/4.0/}{Creative Commons} Lizenz veröffentlicht.

\textcolor{red}{Ich erhebe keinen Anspruch auf Vollständigkeit oder Richtigkeit. Falls ihr Fehler findet oder etwas fehlt, dann meldet euch bitte über den Emailkontakt.}

\tableofcontents


\newpage



\section{Aufgabe 1}


\subsection*{a)}

Wir müssen uns klarmachen, das die Bildweite hier $b = -S_0$ ist:

\begin{align*}
\frac{1}{f} &= \frac{1}{g} + \frac{1}{b} = \frac{1}{g} - \frac{1}{S_0} \\
V &= - \frac{b}{g} = - \frac{-S_0}{g} = S_0 \cdot \left( \frac{1}{f} + \frac{1}{S_0} \right) = \frac{S_0}{f} + 1
\end{align*}


\subsection*{b)}

Dioptrin ist einfach der Kehrwert der Brennweite, also können wir folgendermaßen rechnen:

\begin{align*}
f &= \frac{1}{D} = \frac{1}{20} = \unit[5]{cm} \\
V &= \frac{S_0}{f} + 1 = \frac{30}{5} + 1 = 7
\end{align*}


\section{Aufgabe 2}


Wir legen die Größen fest:

\begin{align*}
G = \unit[2]{m} \qquad g &= \unit[50]{m} \qquad B = \unit[36]{mm} \\
\hfill \\
\left| \frac{B}{G} \right| &= \frac{b}{g} \\
\Leftrightarrow b &= g \cdot \frac{B}{G} = 50 \cdot \frac{36 \cdot 10^{-3}}{2} = \unit[0,9]{m} \\
\hfill \\
f &= \frac{1}{\frac{1}{50} + \frac{1}{0,9}} = \unit[0,88]{m}
\end{align*}


\newpage

\section{Aufgabe 3}

Wir haben die Situation auf dem Bild:

\begin{figure}[h]
	\centering
	\includegraphics[scale=0.15]{A3_1.jpg}
\end{figure}


\begin{align*}
g &= \unit[23]{cm} \qquad b = \unit[-48]{cm} \\
\hfill \\
f &= \frac{1}{\frac{1}{23} - \frac{1}{48}} = \unit[0,44]{m} \\
D &= \frac{1}{f} = \frac{1}{0,44} = \unit[2,27]{dpt}
\end{align*}


\section{Aufgabe 4}

Wir wissen, das gilt:

\begin{align*}
\frac{f_{Ob}}{f_{Ok}} &= 7 \\
\Leftrightarrow f_{Ob} &= 7 \cdot f_{Ok} \\
\hfill \\
f_{Ob} + f_{Ok} &= 32 = 7 \cdot f_{Ok} + f_{Ok} \\
\Rightarrow f_{Ok} &= \unit[4]{cm} \qquad f_{Ob} = \unit[28]{cm}
\end{align*}

\newpage

\section{Aufgabe 5}

Die Situation ist wie auf dem Bild dargestellt:

\begin{figure}[h]
	\centering
	\includegraphics[scale=0.12]{A5_1.jpg}
\end{figure}

\subsection*{a)}


\begin{align*}
\frac{1}{g} &= \frac{1}{f} - \frac{1}{b} = \frac{1}{1,7} - \frac{1}{16 + 1,7} \\
\Leftrightarrow g &= \unit[1,8]{cm}
\end{align*}


\subsection*{b)}

\begin{align*}
V_M &= - \frac{t}{f_{Ob}} \cdot \frac{S_0}{f_{Ok}} = - \frac{16}{1,7} \cdot \frac{25}{5,1} = - 46,1
\end{align*}


\section{Aufgabe 6}

Die Situation ist wie auf dem Bild dargestellt:

\begin{figure}[h]
	\centering
	\includegraphics[scale=0.12]{A6_1.jpg}
\end{figure}

\hfill \\

Vom Teleskop wissen wir zudem das gilt $V_T = \frac{\theta_{Ok}}{\theta_{Ob}}$.

\begin{align*}
\intertext{Für das Objektiv gilt:}
\tan \left( \theta_{Ob} \right) &= \frac{h}{f_{Ob}} \\
\Rightarrow \theta_{Ob} \approx \frac{h}{f_{Ob}}
\intertext{Das gleiche giltfür das Okular:}
\tan \left( \theta_{Ok} \right) &= \frac{-h}{f_{Ok}} \\
\Rightarrow \theta_{Ok} \approx \frac{-h}{f_{Ok}}
\intertext{Die Vergrößerung ist dann:}
V_T &= \frac{ \frac{-h}{f_{Ok}}}{\frac{h}{f_{Ob}}} = - \frac{f_{Ob}}{f_{Ok}} > 0
\end{align*}

Das Bild ist damit aufrecht und virtuell.












\end{document}