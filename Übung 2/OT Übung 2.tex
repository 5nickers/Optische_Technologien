\input{header.tex}


\begin{document}

\maketitle

Dieser Text ist unter dieser \href{http://creativecommons.org/licenses/by-nc-sa/4.0/}{Creative Commons} Lizenz veröffentlicht.

\textcolor{red}{Ich erhebe keinen Anspruch auf Vollständigkeit oder Richtigkeit. Falls ihr Fehler findet oder etwas fehlt, dann meldet euch bitte über den Emailkontakt.}

\tableofcontents


\newpage



\section{Aufgabe 1}

Bild einfügen!!!

Wir rechnen mit Vielfachen des Radius R! Zunächst bestimmen wir die Geradengleichung für die dunkelblaue Gerade:

\begin{align*}
y_1 &= a \cdot x \qquad \text{mit} \qquad G = 0,1 \qquad \text{und} \qquad \overline{CG} = \frac{1}{3} \\
a &= \frac{G}{\overline{CG}} = \frac{0,1}{\frac{1}{3}} = 0,3 \\
\Rightarrow y_1 &= 0,3 \cdot x
\intertext{Nun stellen wir einen Zusammenhang zwischen $\alpha$ und $\beta$ her:}
n \cdot \sin(\alpha) &= \sin(\beta) \\
n \cdot \alpha &= \beta = \underbrace{1,5}_{Glas} \cdot \alpha
\intertext{Nun folgt die Gleichung für dir hellblaue Gerade. Mit der x-Achse schließt die Gerade den Winkel $\beta - \alpha$. Im weiteren ist $\beta - \alpha$ eine Steigung und kein Winkel!!}
\beta - \alpha &= 0,5 \cdot \alpha \\
\alpha &= \frac{G}{R} = \frac{0,1}{1} = 0,1 \\
\Rightarrow \beta - \alpha = 0,5 \cdot 0,1 = 0,05
\intertext{Wir schauen uns nun den Strahlengang an und stellen mit der Steigung und dem markanten Punkt P die Geradengleichung auf:}
y_2 &= 0,1 + 0,05 \cdot (x - 1) = 0,05x + 0,15
\intertext{Um den Schnittpunkt zu bestimmen, setzen wir beide Gleichungen gleich:}
y_1 &= y_2 \\
\Leftrightarrow 0,3x &= 0,05x - 0,15 \\
\Leftrightarrow 0,25x &= 0,15 \\
\Leftrightarrow x &= \frac{3}{5} = 0,6
\intertext{Über das Verhältnis von Bild zu Gegenstand berechnen wir die Bildgröße:}
\frac{B}{G} = \frac{0,6}{0,3} = 1,8
\end{align*}

Der Gegenstand wir also um den Faktor 1,8 vergrößert.


\section{Aufgabe 2}

\begin{align*}
\frac{1}{f} &= \left( 1,52 - 1 \right) \cdot \left( \frac{1}{50} - \frac{1}{-50} \right) = \frac{0,52}{25} \\
\Leftrightarrow f &= \frac{25}{0,52} = \unit[48,1]{mm}
\end{align*}


\section{Aufgabe 3}

\begin{align*}
\frac{1}{f} &= \left( 1,52 - 1 \right) \cdot \left( \frac{1}{100} - \frac{1}{-\infty} \right) = \frac{0,52}{100} \\
\Leftrightarrow f &= \frac{100}{0,52} = \unit[192,3]{mm}
\end{align*}


\section{Aufgabe 4}

Bild einfügen!!!!


Für die blaue Strecke $\half t$ gilt:

\begin{align*}
\half t &= 50 - \sqrt{50^2 - 25^2} \\
\Leftrightarrow t &= \unit[13,4]{mm}
\intertext{Dies können wir in die Gleichung für dicke Linsen einsetzen:}
\frac{1}{f} &= \left( 1,52 - 1 \right) \cdot \left( \frac{1}{50} - \frac{1}{-50} + \frac{0,52 \cdot 13,4}{1,52 \cdot 50 \cdot -50} \right) \\
\Leftrightarrow f &= \unit[50,4]{mm}
\end{align*}


\section{Aufgabe 5}

Bei einer planen Fläche auf einer Seite, wird einer der Radien $\infty$ und damit wird der Korrekturterm der dicken Linse $0$ und fällt weg und wir erhalten die selbe Brennweite wie in Aufgabe 2.

\end{document}