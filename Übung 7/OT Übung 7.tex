\input{header.tex}


\begin{document}

\maketitle

Dieser Text ist unter dieser \href{http://creativecommons.org/licenses/by-nc-sa/4.0/}{Creative Commons} Lizenz veröffentlicht.

\textcolor{red}{Ich erhebe keinen Anspruch auf Vollständigkeit oder Richtigkeit. Falls ihr Fehler findet oder etwas fehlt, dann meldet euch bitte über den Emailkontakt.}

\tableofcontents


\newpage



\section{Aufgabe 1}


\begin{align*}
I \sim \frac{\sin^2\left( \frac{N \phi}{2} \right)}{\sin\left( \frac{\phi}{2} \right)} \underset{\phi \rightarrow 0, \ \sin(x) \approx x}{\longrightarrow} \frac{\left( \frac{N \phi}{2} \right)^2}{\frac{\phi}{2}} = N^2 
\end{align*}


\section{Aufgabe 2}

nicht lösbar


\section{Aufgabe 3}


\begin{figure}[h]
	\centering
	\includegraphics[scale=0.6]{A3_1.jpg}	
\end{figure}

\begin{align*}
I(\theta) &= \frac{\sin^2\left( \frac{\pi b \sin(\theta)}{\lambda} \right)}{\left( \frac{\pi b \sin(\theta)}{\lambda} \right)^2}
\intertext{Im ersten Minimum gilt:}
\pi &= \frac{\pi b \sin(\theta)}{\lambda}
\intertext{Da wir kleine Winkel betrachten gilt auch:}
\tan(\theta) &= \theta = \sin(\theta) = \frac{\lambda}{b}
\intertext{Wir definieren $\Delta x$ als den Abstand vom zentralen Maximum zum ersten Minumum:}
\Delta x &= L \cdot \frac{\lambda}{b}
\end{align*}


\newpage

\section{Aufgabe 4}

Die Intensität ist hier:

\begin{align*}
I(\theta) \sim \left[ \frac{2 j_1 \cdot \left( \pi d \cdot \frac{\sin(\theta)}{\lambda} \right)}{\pi d \frac{\sin(\theta)}{\lambda}} \right]^2
\intertext{Der erste dunkle Ring entspricht nun der ersten Nullstelle von $j_1$. Aus der Vorlesund wissen wir das dies bei $x = 1,22 \pi$ der Fall ist:}
1,22 \pi &= \pi d \cdot \frac{\sin(\theta)}{\lambda} \\
\Leftrightarrow \sin(\theta) &= 1,22 \cdot \frac{\lambda}{d} \approx \theta
\end{align*}

Dabei ist $d$ der Durchmesser des Lochs. Ab hier geht es dann weiter wie in A3.



\section{Aufgabe 5}

Das Beugungsbild sieht ungefähr so aus:

\begin{figure}[h]
	\centering
	\includegraphics[scale=0.43]{A5_1.jpg}	
\end{figure}


Dabei gelten die folgenden Zusammenhänge:

\begin{align*}
\intertext{Maxima und in Phase}
d \cdot sin(\theta) &= N \cdot \lambda 
\intertext{Minima:}
d \cdot sin(\theta) &= \left( N + \half \right) \cdot \lambda 
\end{align*}





















\end{document}