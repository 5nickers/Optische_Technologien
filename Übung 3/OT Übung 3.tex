\documentclass[11pt]{scrartcl}
\usepackage[T1]{fontenc}
\usepackage[a4paper, left=3cm, right=2cm, top=2cm, bottom=2cm]{geometry}
\usepackage[activate]{pdfcprot}
\usepackage[ngerman]{babel}
\usepackage[parfill]{parskip}
\usepackage[utf8]{inputenc}
\usepackage{kurier}
\usepackage{amsmath}
\usepackage{amssymb}
\usepackage{xcolor}
\usepackage{epstopdf}
\usepackage{txfonts}
\usepackage{fancyhdr}
\usepackage{graphicx}
\usepackage{prettyref}
\usepackage{hyperref}
\usepackage{eurosym}
\usepackage{setspace}
\usepackage{units}
\usepackage{eso-pic,graphicx}
\usepackage{icomma}
\usepackage{pdfpages}

\definecolor{darkblue}{rgb}{0,0,.5}
\hypersetup{pdftex=true, colorlinks=true, breaklinks=false, linkcolor=black, menucolor=black, pagecolor=black, urlcolor=darkblue}



\setlength{\columnsep}{2cm}


\newcommand{\arcsinh}{\mathrm{arcsinh}}
\newcommand{\asinh}{\mathrm{arcsinh}}
\newcommand{\ergebnis}{\textcolor{red}{\mathrm{Ergebnis}}}
\newcommand{\fehlt}{\textcolor{red}{Hier fehlen noch Inhalte.}}
\newcommand{\betanotice}{\textcolor{red}{Diese Aufgaben sind noch nicht in der Übung kontrolliert worden. Es sind lediglich meine Überlegungen und Lösungsansätze zu den Aufgaben. Es können Fehler enthalten sein!!! Das Dokument wird fortwährend aktualisiert und erst wenn das \textcolor{black}{beta} aus dem Dateinamen verschwindet ist es endgültig.}}
\newcommand{\half}{\frac{1}{2}}
\renewcommand{\d}{\, \mathrm d}
\newcommand{\punkte}{\textcolor{white}{xxxxx}}
\newcommand{\p}{\, \partial}
\newcommand{\dd}[1]{\item[#1] \hfill \\}

\renewcommand{\familydefault}{\sfdefault}
\renewcommand\thesection{}
\renewcommand\thesubsection{}
\renewcommand\thesubsubsection{}


\newcommand{\themodul}{Optische Technologie}
\newcommand{\thetutor}{Prof. Rateike}
\newcommand{\theuebung}{Übung 2}

\pagestyle{fancy}
\fancyhead[L]{\footnotesize{C. Hansen}}
\chead{\thepage}
\rhead{}
\lfoot{}
\cfoot{}
\rfoot{}

\title{\themodul{}, \theuebung{}, \thetutor}


\author{Christoph Hansen \\ {\small \href{mailto:chris@university-material.de}{chris@university-material.de}} }

\date{}


\begin{document}

\maketitle

Dieser Text ist unter dieser \href{http://creativecommons.org/licenses/by-nc-sa/4.0/}{Creative Commons} Lizenz veröffentlicht.

\textcolor{red}{Ich erhebe keinen Anspruch auf Vollständigkeit oder Richtigkeit. Falls ihr Fehler findet oder etwas fehlt, dann meldet euch bitte über den Emailkontakt.}

\tableofcontents


\newpage



\section{Aufgabe 1}

In dieser Aufgabe ist $h$ eine Variable. In Abhängigkeit der Variable $h$ bestimmen wir die Position $x_2$. Wir lassen $h$ dabei von $0 - 0,7$ laufen. \\

Zunächst bestimmen wir $h$ und $x_1$:

\begin{align*}
\sin(\alpha) &= h \Leftrightarrow \alpha = \arcsin(h) \\
x_1 &= \cos(\alpha)
\intertext{Zudem können wir $\gamma$ ablesen:}
\gamma &= \beta - \alpha 
\intertext{Über das Brechungsgesetz bestimmen wir nun $\beta$:}
n \cdot \sin(\alpha) &= \sin(\beta) \\
\Leftrightarrow \beta &= \arcsin(\sin(\alpha) \cdot n) = \arcsin(h \cdot n)
\intertext{Wir berechnen die Strecke zwischen $x_1$ und $x_2$:}
\tan(\gamma) &= \frac{h}{x_2 - x_1} \Leftrightarrow x_2 - x_1 = \frac{h}{\tan(\gamma)}
\end{align*}


Man generiert sich jetzt eine Menge Werte für $x_1, \alpha, \gamma, \beta$ berechnet daraus $x_2$. Das packt man in einen Graphen indem man auf der x-Achse die Höhe h aufträgt und auf der y-Achse $x_2$. Da erkennt man wie stark $x_2$ sich verändert.



\section{Aufgabe 2}


Diese Aufgabe funktioniert ähnlich wie die Aufgabe 1, aber es sind zwei Schritte nötig:




















\end{document}