\input{header.tex}


\begin{document}

\maketitle

Dieser Text ist unter dieser \href{http://creativecommons.org/licenses/by-nc-sa/4.0/}{Creative Commons} Lizenz veröffentlicht.

\textcolor{red}{Ich erhebe keinen Anspruch auf Vollständigkeit oder Richtigkeit. Falls ihr Fehler findet oder etwas fehlt, dann meldet euch bitte über den Emailkontakt.}

\tableofcontents


\newpage



\section{Aufgabe 1}

In dieser Aufgabe ist $h$ eine Variable. In Abhängigkeit der Variable $h$ bestimmen wir die Position $x_2$. Wir lassen $h$ dabei von $0 - 0,7$ laufen. \\

Zunächst bestimmen wir $h$ und $x_1$:

\begin{align*}
\sin(\alpha) &= h \Leftrightarrow \alpha = \arcsin(h) \\
x_1 &= \cos(\alpha)
\intertext{Zudem können wir $\gamma$ ablesen:}
\gamma &= \beta - \alpha 
\intertext{Über das Brechungsgesetz bestimmen wir nun $\beta$:}
n \cdot \sin(\alpha) &= \sin(\beta) \\
\Leftrightarrow \beta &= \arcsin(\sin(\alpha) \cdot n) = \arcsin(h \cdot n)
\intertext{Wir berechnen die Strecke zwischen $x_1$ und $x_2$:}
\tan(\gamma) &= \frac{h}{x_2 - x_1} \Leftrightarrow x_2 - x_1 = \frac{h}{\tan(\gamma)}
\end{align*}


Man generiert sich jetzt eine Menge Werte für $x_1, \alpha, \gamma, \beta$ berechnet daraus $x_2$. Das packt man in einen Graphen indem man auf der x-Achse die Höhe h aufträgt und auf der y-Achse $x_2$. Da erkennt man wie stark $x_2$ sich verändert.



\section{Aufgabe 2}


Diese Aufgabe funktioniert ähnlich wie die Aufgabe 1, aber es sind zwei Schritte nötig:




















\end{document}