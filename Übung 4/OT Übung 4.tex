\documentclass[11pt]{scrartcl}
\usepackage[T1]{fontenc}
\usepackage[a4paper, left=3cm, right=2cm, top=2cm, bottom=2cm]{geometry}
\usepackage[activate]{pdfcprot}
\usepackage[ngerman]{babel}
\usepackage[parfill]{parskip}
\usepackage[utf8]{inputenc}
\usepackage{kurier}
\usepackage{amsmath}
\usepackage{amssymb}
\usepackage{xcolor}
\usepackage{epstopdf}
\usepackage{txfonts}
\usepackage{fancyhdr}
\usepackage{graphicx}
\usepackage{prettyref}
\usepackage{hyperref}
\usepackage{eurosym}
\usepackage{setspace}
\usepackage{units}
\usepackage{eso-pic,graphicx}
\usepackage{icomma}
\usepackage{pdfpages}

\definecolor{darkblue}{rgb}{0,0,.5}
\hypersetup{pdftex=true, colorlinks=true, breaklinks=false, linkcolor=black, menucolor=black, pagecolor=black, urlcolor=darkblue}



\setlength{\columnsep}{2cm}


\newcommand{\arcsinh}{\mathrm{arcsinh}}
\newcommand{\asinh}{\mathrm{arcsinh}}
\newcommand{\ergebnis}{\textcolor{red}{\mathrm{Ergebnis}}}
\newcommand{\fehlt}{\textcolor{red}{Hier fehlen noch Inhalte.}}
\newcommand{\betanotice}{\textcolor{red}{Diese Aufgaben sind noch nicht in der Übung kontrolliert worden. Es sind lediglich meine Überlegungen und Lösungsansätze zu den Aufgaben. Es können Fehler enthalten sein!!! Das Dokument wird fortwährend aktualisiert und erst wenn das \textcolor{black}{beta} aus dem Dateinamen verschwindet ist es endgültig.}}
\newcommand{\half}{\frac{1}{2}}
\renewcommand{\d}{\, \mathrm d}
\newcommand{\punkte}{\textcolor{white}{xxxxx}}
\newcommand{\p}{\, \partial}
\newcommand{\dd}[1]{\item[#1] \hfill \\}

\renewcommand{\familydefault}{\sfdefault}
\renewcommand\thesection{}
\renewcommand\thesubsection{}
\renewcommand\thesubsubsection{}


\newcommand{\themodul}{Optische Technologie}
\newcommand{\thetutor}{Prof. Rateike}
\newcommand{\theuebung}{Übung 2}

\pagestyle{fancy}
\fancyhead[L]{\footnotesize{C. Hansen}}
\chead{\thepage}
\rhead{}
\lfoot{}
\cfoot{}
\rfoot{}

\title{\themodul{}, \theuebung{}, \thetutor}


\author{Christoph Hansen \\ {\small \href{mailto:chris@university-material.de}{chris@university-material.de}} }

\date{}


\begin{document}

\maketitle

Dieser Text ist unter dieser \href{http://creativecommons.org/licenses/by-nc-sa/4.0/}{Creative Commons} Lizenz veröffentlicht.

\textcolor{red}{Ich erhebe keinen Anspruch auf Vollständigkeit oder Richtigkeit. Falls ihr Fehler findet oder etwas fehlt, dann meldet euch bitte über den Emailkontakt.}

\tableofcontents


\newpage



\section{Aufgabe 1}

\subsection*{a)}

Die Transmission ist beschrieben durch:

\begin{align*}
T_\perp &= e^{-\mu_\perp \cdot d} \\
T_\parallel &= e^{-\mu_\parallel \cdot d} \\
\end{align*}


\subsection*{b)}

Man kann die Folie dicker machen, allerdings geht dann auch die Transmission der anderen Polarisation runter. Man muss also einen guten Mittelweg finden.


\subsection*{c)}

\begin{itemize}
	\item[1)] wenn man zwei Polarisatoren kreuzt, dann sollte die Auslöschung bei $E > 5$ liegen. \\ $E = -\log(T)$
	\item[2)] parallel muss die maximale Transmission durch die Polarisatoren kommen
\end{itemize}

Gute Polfilter lassen $\approx \unit[40]{\%}$ des Lichtes durch. Ideal wären $\approx \unit[50]{\%}$, aber wegen Reflexionsverlusten ist das nicht möglich.



\section{Aufgabe 2}

Wir nehmen einen Laser mit der Wellenlänge $\lambda = \unit[633]{nm}$. Dann können wir recht einfach die Dick berechnen:

\begin{align*}
d \cdot \Delta n &= \frac{\lambda}{4} \\
\Leftrightarrow d &= \frac{\lambda}{4 \cdot |n_o - n_e|} = \frac{633 \cdot 10^{-9}}{4 \cdot 0,0091} = \unit[17,39]{\mu m}
\intertext{Da eine solche Dicke schwer zu realisieren ist, behilft man sich mit einem Trick:}
d_{ges} &= d_{\lambda/4} + m \cdot d_{\lambda}
\end{align*}

Das heißt man nimmt die Dicke der $\lambda/4$ Platte und macht sie um ganzzahlige Lambdadicken dicker, da ganzzahlige nichts am Plattenverhalten ändern. Dabei kommen natürlich Ungenauigkeiten hinzu, weshalb es eine teurer Alternative gibt. \\
Die Alternative verwendet eine Platte mit  $d_{ges} = d_{\lambda/4} + m \cdot d_{\lambda}$ und eine mit $d_{ges} = - m \cdot d_{\lambda}$. Dabei wird der \grqq Überhang\grqq \  der ersten Platte durch die zweite kompensiert und man hat reine $\lambda/4$.


\section{Aufgabe 3}

Aus der Vorlesung wissen wir, das bei $\theta = \unit[0]{^\circ}$ zirkular polarisiertes Licht entsteht und es bei $\theta = \unit[90]{^\circ}$ maximal elliptisch ist bzw. der außerordentliche Anteil am größten wird.








\end{document}