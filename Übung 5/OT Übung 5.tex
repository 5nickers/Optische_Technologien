\documentclass[11pt]{scrartcl}
\usepackage[T1]{fontenc}
\usepackage[a4paper, left=3cm, right=2cm, top=2cm, bottom=2cm]{geometry}
\usepackage[activate]{pdfcprot}
\usepackage[ngerman]{babel}
\usepackage[parfill]{parskip}
\usepackage[utf8]{inputenc}
\usepackage{kurier}
\usepackage{amsmath}
\usepackage{amssymb}
\usepackage{xcolor}
\usepackage{epstopdf}
\usepackage{txfonts}
\usepackage{fancyhdr}
\usepackage{graphicx}
\usepackage{prettyref}
\usepackage{hyperref}
\usepackage{eurosym}
\usepackage{setspace}
\usepackage{units}
\usepackage{eso-pic,graphicx}
\usepackage{icomma}
\usepackage{pdfpages}

\definecolor{darkblue}{rgb}{0,0,.5}
\hypersetup{pdftex=true, colorlinks=true, breaklinks=false, linkcolor=black, menucolor=black, pagecolor=black, urlcolor=darkblue}



\setlength{\columnsep}{2cm}


\newcommand{\arcsinh}{\mathrm{arcsinh}}
\newcommand{\asinh}{\mathrm{arcsinh}}
\newcommand{\ergebnis}{\textcolor{red}{\mathrm{Ergebnis}}}
\newcommand{\fehlt}{\textcolor{red}{Hier fehlen noch Inhalte.}}
\newcommand{\betanotice}{\textcolor{red}{Diese Aufgaben sind noch nicht in der Übung kontrolliert worden. Es sind lediglich meine Überlegungen und Lösungsansätze zu den Aufgaben. Es können Fehler enthalten sein!!! Das Dokument wird fortwährend aktualisiert und erst wenn das \textcolor{black}{beta} aus dem Dateinamen verschwindet ist es endgültig.}}
\newcommand{\half}{\frac{1}{2}}
\renewcommand{\d}{\, \mathrm d}
\newcommand{\punkte}{\textcolor{white}{xxxxx}}
\newcommand{\p}{\, \partial}
\newcommand{\dd}[1]{\item[#1] \hfill \\}

\renewcommand{\familydefault}{\sfdefault}
\renewcommand\thesection{}
\renewcommand\thesubsection{}
\renewcommand\thesubsubsection{}


\newcommand{\themodul}{Optische Technologie}
\newcommand{\thetutor}{Prof. Rateike}
\newcommand{\theuebung}{Übung 2}

\pagestyle{fancy}
\fancyhead[L]{\footnotesize{C. Hansen}}
\chead{\thepage}
\rhead{}
\lfoot{}
\cfoot{}
\rfoot{}

\title{\themodul{}, \theuebung{}, \thetutor}


\author{Christoph Hansen \\ {\small \href{mailto:chris@university-material.de}{chris@university-material.de}} }

\date{}


\begin{document}

\maketitle

Dieser Text ist unter dieser \href{http://creativecommons.org/licenses/by-nc-sa/4.0/}{Creative Commons} Lizenz veröffentlicht.

\textcolor{red}{Ich erhebe keinen Anspruch auf Vollständigkeit oder Richtigkeit. Falls ihr Fehler findet oder etwas fehlt, dann meldet euch bitte über den Emailkontakt.}

\tableofcontents


\newpage



\section{Aufgabe 1}

\subsection*{1)}

\begin{align*}
\half 
\left[
\begin{matrix}
1 & 1 & 0 & 0 \\ 
1 & 1 & 0 & 0 \\ 
0 & 0 & 0 & 0 \\ 
0 & 0 & 0 & 0
\end{matrix} 
\right]
\cdot
\left(
\begin{matrix}
1 \\ 
1 \\ 
0 \\ 
0
\end{matrix} 
\right)
=
\half
\left(
\begin{matrix}
2 \\ 
2 \\ 
0 \\ 
0
\end{matrix} 
\right)
=
\left(
\begin{matrix}
1 \\ 
1 \\ 
0 \\ 
0
\end{matrix} 
\right)
\end{align*}


\subsection*{2)}

\begin{align*}
\half 
\left[
\begin{matrix}
1 & -1 & 0 & 0 \\ 
-1 & 1 & 0 & 0 \\ 
0 & 0 & 0 & 0 \\ 
0 & 0 & 0 & 0
\end{matrix} 
\right]
\cdot
\left(
\begin{matrix}
1 \\ 
1 \\ 
0 \\ 
0
\end{matrix} 
\right)
=
\left(
\begin{matrix}
0 \\ 
0 \\ 
0 \\ 
0
\end{matrix} 
\right)
\end{align*}


\subsection*{3)}

\begin{align*}
\half 
\left[
\begin{matrix}
1 & 0 & 1 & 0 \\ 
0 & 0 & 0 & 0 \\ 
1 & 0 & 1 & 0 \\ 
0 & 0 & 0 & 0
\end{matrix} 
\right]
\cdot
\left(
\begin{matrix}
1 \\ 
1 \\ 
0 \\ 
0
\end{matrix} 
\right)
=
\half
\left(
\begin{matrix}
1 \\ 
0 \\ 
1 \\ 
0
\end{matrix} 
\right)
=
\left(
\begin{matrix}
\half \\ 
0 \\ 
\half \\ 
0
\end{matrix} 
\right)
\end{align*}


\subsection*{4)}

\begin{align*}
\left[
\begin{matrix}
1 & 0 & 0 & 0 \\ 
0 & 1 & 0 & 0 \\ 
0 & 0 & 0 & -1 \\ 
0 & 0 & 1 & 0
\end{matrix} 
\right]
\cdot
\left(
\begin{matrix}
1 \\ 
(-)1 \\ 
0 \\ 
0
\end{matrix} 
\right)
=
\left(
\begin{matrix}
1 \\ 
(-)1 \\ 
0 \\ 
0
\end{matrix} 
\right)
\end{align*}

In Klammern steht was passiert wenn die schnelle Achse horizontal ist.

\subsection*{5)}

\begin{align*} 
\left[
\begin{matrix}
1 & 0 & 0 & 0 \\ 
0 & 1 & 0 & 0 \\ 
0 & 0 & 0 & -1 \\ 
0 & 0 & 1 & 0
\end{matrix} 
\right]
\cdot
\left(
\begin{matrix}
1 \\ 
0 \\ 
(-)1 \\ 
0
\end{matrix} 
\right)
=
\left(
\begin{matrix}
1 \\ 
0 \\ 
0 \\ 
(-)1
\end{matrix} 
\right)
\end{align*}

In Klammern steht was bei $\unit[-45]{^\circ}$ p-polarisiertem Licht rauskommt.

\subsection*{6)}

\begin{align*}
\half 
\left[
\begin{matrix}
1 & 0 & 0 & 0 \\ 
0 & 1 & 0 & 0 \\ 
0 & 0 & 0 & -1 \\ 
0 & 0 & 1 & 0
\end{matrix} 
\right]
\cdot
\left(
\begin{matrix}
1 \\ 
0 \\ 
0 \\ 
(-)1
\end{matrix} 
\right)
=
\half
\left(
\begin{matrix}
1 \\ 
1 \\ 
0 \\ 
0
\end{matrix} 
\right)
=
\left(
\begin{matrix}
\half \\ 
\half \\ 
0 \\ 
0
\end{matrix} 
\right)
\end{align*}

Es die Drehrichtung des zirkular polarisierten Lichtes hat also wie zu erwarten keinen Einfluss. 



\end{document}