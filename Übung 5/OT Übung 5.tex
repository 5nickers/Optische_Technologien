\input{header.tex}


\begin{document}

\maketitle

Dieser Text ist unter dieser \href{http://creativecommons.org/licenses/by-nc-sa/4.0/}{Creative Commons} Lizenz veröffentlicht.

\textcolor{red}{Ich erhebe keinen Anspruch auf Vollständigkeit oder Richtigkeit. Falls ihr Fehler findet oder etwas fehlt, dann meldet euch bitte über den Emailkontakt.}

\tableofcontents


\newpage



\section{Aufgabe 1}

\subsection*{1)}

\begin{align*}
\half 
\left[
\begin{matrix}
1 & 1 & 0 & 0 \\ 
1 & 1 & 0 & 0 \\ 
0 & 0 & 0 & 0 \\ 
0 & 0 & 0 & 0
\end{matrix} 
\right]
\cdot
\left(
\begin{matrix}
1 \\ 
1 \\ 
0 \\ 
0
\end{matrix} 
\right)
=
\half
\left(
\begin{matrix}
2 \\ 
2 \\ 
0 \\ 
0
\end{matrix} 
\right)
=
\left(
\begin{matrix}
1 \\ 
1 \\ 
0 \\ 
0
\end{matrix} 
\right)
\end{align*}


\subsection*{2)}

\begin{align*}
\half 
\left[
\begin{matrix}
1 & -1 & 0 & 0 \\ 
-1 & 1 & 0 & 0 \\ 
0 & 0 & 0 & 0 \\ 
0 & 0 & 0 & 0
\end{matrix} 
\right]
\cdot
\left(
\begin{matrix}
1 \\ 
1 \\ 
0 \\ 
0
\end{matrix} 
\right)
=
\left(
\begin{matrix}
0 \\ 
0 \\ 
0 \\ 
0
\end{matrix} 
\right)
\end{align*}


\subsection*{3)}

\begin{align*}
\half 
\left[
\begin{matrix}
1 & 0 & 1 & 0 \\ 
0 & 0 & 0 & 0 \\ 
1 & 0 & 1 & 0 \\ 
0 & 0 & 0 & 0
\end{matrix} 
\right]
\cdot
\left(
\begin{matrix}
1 \\ 
1 \\ 
0 \\ 
0
\end{matrix} 
\right)
=
\half
\left(
\begin{matrix}
1 \\ 
0 \\ 
1 \\ 
0
\end{matrix} 
\right)
=
\left(
\begin{matrix}
\half \\ 
0 \\ 
\half \\ 
0
\end{matrix} 
\right)
\end{align*}


\subsection*{4)}

\begin{align*}
\left[
\begin{matrix}
1 & 0 & 0 & 0 \\ 
0 & 1 & 0 & 0 \\ 
0 & 0 & 0 & -1 \\ 
0 & 0 & 1 & 0
\end{matrix} 
\right]
\cdot
\left(
\begin{matrix}
1 \\ 
(-)1 \\ 
0 \\ 
0
\end{matrix} 
\right)
=
\left(
\begin{matrix}
1 \\ 
(-)1 \\ 
0 \\ 
0
\end{matrix} 
\right)
\end{align*}

In Klammern steht was passiert wenn die schnelle Achse horizontal ist.

\subsection*{5)}

\begin{align*} 
\left[
\begin{matrix}
1 & 0 & 0 & 0 \\ 
0 & 1 & 0 & 0 \\ 
0 & 0 & 0 & -1 \\ 
0 & 0 & 1 & 0
\end{matrix} 
\right]
\cdot
\left(
\begin{matrix}
1 \\ 
0 \\ 
(-)1 \\ 
0
\end{matrix} 
\right)
=
\left(
\begin{matrix}
1 \\ 
0 \\ 
0 \\ 
(-)1
\end{matrix} 
\right)
\end{align*}

In Klammern steht was bei $\unit[-45]{^\circ}$ p-polarisiertem Licht rauskommt.

\subsection*{6)}

\begin{align*}
\half 
\left[
\begin{matrix}
1 & 0 & 0 & 0 \\ 
0 & 1 & 0 & 0 \\ 
0 & 0 & 0 & -1 \\ 
0 & 0 & 1 & 0
\end{matrix} 
\right]
\cdot
\left(
\begin{matrix}
1 \\ 
0 \\ 
0 \\ 
(-)1
\end{matrix} 
\right)
=
\half
\left(
\begin{matrix}
1 \\ 
1 \\ 
0 \\ 
0
\end{matrix} 
\right)
=
\left(
\begin{matrix}
\half \\ 
\half \\ 
0 \\ 
0
\end{matrix} 
\right)
\end{align*}

Es die Drehrichtung des zirkular polarisierten Lichtes hat also wie zu erwarten keinen Einfluss. 



\end{document}